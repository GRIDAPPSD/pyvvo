\usetikzlibrary{shapes.geometric, arrows.meta, fit, calc, positioning, automata, decorations.pathreplacing}

% When we exceed 26 characters...
% https://tex.stackexchange.com/a/269559/208656
\usepackage{alphalph}

% Declare the layers
% https://tex.stackexchange.com/a/75498/208656
\pgfdeclarelayer{background}
\pgfsetlayers{background,main}

% tikzset for flowcharts.
% https://tex.stackexchange.com/a/342227/208656
% Reference for label shifting in style:
% https://tex.stackexchange.com/a/78690/208656
% Reference for curly braces:
% https://tex.stackexchange.com/a/20887/208656
\tikzset{flowchart/.style = {
	base/.style = {
		draw=black,
		very thick,
		inner sep=10pt,
		outer sep=0pt,
		text width=10.0cm,
		minimum height=0.8cm,
		align=flush center,
	},
	labelshift/.style = {
		prefix after command= {\pgfextra{\tikzset{every label/.style={xshift=0.42cm}}}}
	},
	startstop/.style = {
		base,
		labelshift,
		rectangle,
		rounded corners,
		fill=red!30
	},
	io/.style = {
		base,
		trapezium,
		trapezium left angle=70,
		trapezium right angle=110,
		trapezium stretches=true,
		fill=blue!30,
	},
	process/.style = {
		base,
		labelshift,
		rectangle,
		fill=orange!30
	},
	decision/.style = {
		base,
		text width=3.0cm,
		diamond,
		aspect=1.2,
		fill=green!30
	},
	arrows.meta/.style = {
		very thick,
	 	-{Stealth[]}
	},
	curlybrace/.style = {%
		draw=black,
		decorate,
		very thick,
		decoration={brace, amplitude=20pt, raise=6pt, mirror},
	},
	curlynode/.style = {
		draw=none,
		black,
		midway,
		xshift=2.75cm,
		text width=3cm	
	},
	backgroundbox/.style = {
		draw=none,
		inner sep=0.8cm,
		ultra thick,
		dashed,
		fill opacity=0.4
	}
}}

% Shortcut for inline code:
% https://stackoverflow.com/a/21344989
\newcommand{\code}[1]{\texttt{#1}}

% Label counter for the flow chart.
\newcounter{ac}
\renewcommand{\theac}{\alphalph{\value{ac}}}
\newcommand{\ac}[1]{(\refstepcounter{ac}\alphalph{\value{ac}})\label{#1}}

% To put parentheses around references:
\newcommand{\acref}[1]{(\ref{#1})}

% The following are for extracing coordinates.
% https://tex.stackexchange.com/a/33706/208656
\newdimen\XCoord
\newdimen\YCoord
\newcommand*{\ExtractCoordinate}[1]{\path (#1); \pgfgetlastxy{\XCoord}{\YCoord};}%
\newcommand*{\ExtractCoordinateTwo}[1]{\path (#1); \pgfgetlastxy{\XCoordTwo}{\YCoordTwo};}%

% Dimension and command for labeling boxes.
\newdimen\XLabel
\newcommand{\LabelNode}[1]{\ExtractCoordinate{$(#1.east)$}; \node (#1-label) at (\XLabel, \YCoord) {\ac{flow:#1}};}