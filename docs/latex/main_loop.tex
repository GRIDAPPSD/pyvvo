% This LaTex file's purpose is to draw a flow chart representing PyVVO's 
% main loop, which can be found in app.py.

\documentclass[tikz]{standalone}
\usetikzlibrary{shapes.geometric, arrows.meta, fit, calc, positioning, automata, decorations.pathreplacing}

% Declare the layers
% https://tex.stackexchange.com/a/75498/208656
\pgfdeclarelayer{background}
\pgfsetlayers{background,main}

% tikzset for flowcharts.
% https://tex.stackexchange.com/a/342227/208656
% Reference for label shifting in style:
% https://tex.stackexchange.com/a/78690/208656
% Reference for curly braces:
% https://tex.stackexchange.com/a/20887/208656
\tikzset{flowchart/.style = {
	base/.style = {
		draw=black,
		very thick,
		inner sep=10pt,
		outer sep=0pt,
		text width=10.0cm,
		minimum height=0.8cm,
		align=flush center,
	},
	labelshift/.style = {
		prefix after command= {\pgfextra{\tikzset{every label/.style={xshift=0.42cm}}}}
	},
	startstop/.style = {
		base,
		labelshift,
		rectangle,
		rounded corners,
		fill=red!30
	},
	io/.style = {
		base,
		trapezium,
		trapezium left angle=70,
		trapezium right angle=110,
		trapezium stretches=true,
		fill=blue!30,
	},
	process/.style = {
		base,
		labelshift,
		rectangle,
		fill=orange!30
	},
	decision/.style = {
		base,
		text width=3.0cm,
		diamond,
		aspect=1.2,
		fill=green!30
	},
	arrows.meta/.style = {
		very thick,
	 	-{Stealth[]}
	},
	curlybrace/.style = {%
		draw=black,
		decorate,
		very thick,
		decoration={brace, amplitude=20pt, raise=6pt, mirror},
	},
	curlynode/.style = {
		draw=none,
		black,
		midway,
		xshift=2.75cm,
		text width=3cm	
	},
	backgroundbox/.style = {
		draw=none,
		inner sep=0.8cm,
		ultra thick,
		dashed,
		fill opacity=0.4
	}
}}

\begin{document}

% Shortcut for inline code:
% https://stackoverflow.com/a/21344989
\newcommand{\code}[1]{\texttt{#1}}

% Label counter for the flow chart.
\newcounter{ac}
\renewcommand{\theac}{\alph{ac}}
\newcommand{\ac}[1]{(\refstepcounter{ac}\alph{ac})\label{#1}}

% To put parentheses around references:
\newcommand{\acref}[1]{(\ref{#1})}

% The following are for extracing coordinates.
% https://tex.stackexchange.com/a/33706/208656
\newdimen\XCoord
\newdimen\YCoord
\newcommand*{\ExtractCoordinate}[1]{\path (#1); \pgfgetlastxy{\XCoord}{\YCoord};}%
\newcommand*{\ExtractCoordinateTwo}[1]{\path (#1); \pgfgetlastxy{\XCoordTwo}{\YCoordTwo};}%

% Dimension and command for labeling boxes.
\newdimen\XLabel
\newcommand{\LabelNode}[1]{\ExtractCoordinate{$(#1.east)$}; \node (#1-label) at (\XLabel, \YCoord) {\ac{flow:#1}};}

\begin{tikzpicture}[flowchart, node distance=1.2cm] 
% \begin{tikzpicture}
% , outer sep=2cm
\tikzstyle{every node}=[font=\large]
	% note the difference between below of=   and below=of:
	% https://tex.stackexchange.com/q/9386/208656

	%%%%%%%%%%%%%%%%%%%%%%%%%%%%%%%%%%%%%%%%%%%%%%%%%%%%%%%%%%%
	% NODES
	%%%%%%%%%%%%%%%%%%%%%%%%%%%%%%%%%%%%%%%%%%%%%%%%%%%%%%%%%%%
	\node (start)
	[startstop]
	{Start. \code{run\_pyvvo.py} is invoked as a wrapper to run \code{app.py},
	which contains PyVVO's main loop.};
	% Create a coordinate which will be referenced for all labels.
	\coordinate (start-east) at ($(start.east) + (1.5,0)$);
	\path ($(start-east)$); \pgfgetlastxy{\XLabel}{\YCoord};
	% Label our first node.
	\LabelNode{start-east}
	%
	\node (sim-request)
	[io, below=of start]
	{PyVVO receives simulation ID and full simulation request
	(feeder MRID, start time, \textit{etc.}) from the GridAPPS-D platform.};
	\LabelNode{sim-request}
	%
	\node (init-interfaces)
	[process, below=of sim-request]
	{PyVVO initializes objects for interfacing with the GridAPPS-D platform
	(\textit{e.g.}, \code{SPARQLManager}, \code{PlatformManager}, \textit{etc.}).};
	\LabelNode{init-interfaces}
	%
	\node (pull-device-data)
	[io, below=of init-interfaces]
	{PyVVO obtains relevant device data and measurement MRIDs via SPARQL queries
	(\textit{e.g.}, regulators, capacitors, switches, inverters, DGs).};
	\LabelNode{pull-device-data}
	%
	\node (eq-mgrs)
	[process, below=of pull-device-data]
	{Using data from \acref{flow:pull-device-data}, various
	\code{EquipmentSinglePhase} objects and corresponding
	\code{EquipmentManager} objects are initialized. The 
	\code{EquipmentManager}s will be used to update
	\code{EquipmentSinglePhase} object states (\textit{e.g.},
	switch open/closed) over the course of the simulation.};
	\LabelNode{eq-mgrs}
	%
	\node (subscribe)
	[process, below=of eq-mgrs]
	{PyVVO subscribes to simulation output using a \code{SimOutRouter},
	which also filters output to only include measurements from equipment
	identified in \acref{flow:pull-device-data}. The \code{SimOutRouter} routes
	measurements to the various \code{EquipmentManager}s initialized in
	\acref{flow:eq-mgrs}.};
	\LabelNode{subscribe}
	%
	\node (pull-gld)
	[io, below=of subscribe]
	{PyVVO queries the GridAPPS-D API to pull the base GridLAB-D (\code{.glm}) model for the simulation.
	This model represents the nominal configuration of the feeder and its devices.};
	\LabelNode{pull-gld}
	%
	\node (init-glm-mgr)
	[process, below=of pull-gld]
	{PyVVO initializes a \code{GLMManager} which parses and maps the \code{.glm} model from
	\acref{flow:pull-gld}.};
	\LabelNode{init-glm-mgr};
	%
	\node (pull-load-data)
	[io, below=of init-glm-mgr]
	{\textbf{INCOMPLETE:} PyVVO pulls historic load data from the GridAPPS-D platform.
	Specifically, 15-minute average (different intervals possible/acceptable) voltage, 
	active power, and reactive power for each secondary customer.};
	\LabelNode{pull-load-data}
	%
	\node (pull-weather-data)
	[io, below=of pull-load-data]
	{PyVVO pulls historic weather data corresponding to the same time intervals as the 
	load data pulled in \acref{flow:pull-load-data}.};
	\LabelNode{pull-weather-data}
	%
	\node (create-load-models)
	[process, below=of pull-weather-data]
	{\textbf{INCOMPLETE}: PyVVO uses the historic load data \acref{flow:pull-load-data} and
	weather data \acref{flow:create-load-models} to create predictive ZIP load models for each
	meter.};
	\LabelNode{create-load-models}
	%
	\node (update-glm-mgr)
	[process, below=of create-load-models]
	{The \code{GLMManager}'s internal model is updated with current equipment states
	(\textit{e.g.}, inverter injections, switch states, \textit{etc.}) and load models.};
	\LabelNode{update-glm-mgr}
	%
	\node (init-ga)
	[process, below=of update-glm-mgr]
	{Initialize \code{GA} and \code{GAStopper} objects.
	The \code{GA} object manages the entire genetic algorithm for determining optimal
	capacitor and regulator setpoints, while the \code{GAStopper} object listens for system
	changes that might cause the current genetic algorithm run to be invalid (\textit{e.g.},
	system topology changes) and halts the genetic algorithm if such a change occurs.};
	\LabelNode{init-ga}
	%
	\node (run-ga)
	[process, below=of init-ga]
	{Run the genetic algorithm. The \code{GA} object will initialize a ``population'' of
	``individuals,'' where each individual has regulators and capacitors initialized to 
	different states. Each individual runs a GridLAB-D simulation to determine the 
	resulting value of the objective function, which takes into account voltage violations,
	feeder power factor, energy consumption, \textit{etc.} The best individuals (and some
	random low-performers) will undergo ``crossover'' and ``mutation'' to create new 
	individuals with different device states/settings. This process repeats until convergence,
	resulting in the overall best individual, which represents the best device states/settings
	for the given optimization period.};
	\LabelNode{run-ga}
	%
	\node (send-commands)
	[io, below=of run-ga]
	{The resulting best device states (\textit{e.g.}, regulator tap positions and capacitor states)
	are sent as commands into the GridAPPS-D platform. PyVVO waits for a set amount of time
	to confirm that the commands were implemented as directed.};
	\LabelNode{send-commands}
	%
	\node (check-done)
	[decision, below=of send-commands, label={[yshift=+0.5cm] left:no}, label={[xshift=+0.5cm] below:yes}]
	{Continue?};
	\LabelNode{check-done}
	%
	\node (stop)
	[startstop, below=of check-done]
	{Stop.};
	\LabelNode{stop}
	%%%%%%%%%%%%%%%%%%%%%%%%%%%%%%%%%%%%%%%%%%%%%%%%%%%%%%%%%%%
	% ARROWS
	%%%%%%%%%%%%%%%%%%%%%%%%%%%%%%%%%%%%%%%%%%%%%%%%%%%%%%%%%%%
	% Add layers as per https://tex.stackexchange.com/a/75498/208656
	\begin{pgfonlayer}{main}
		\draw[arrows.meta] (start) -- (sim-request);
		%
		\draw[arrows.meta] (sim-request) -- (init-interfaces);
		%
		\draw[arrows.meta] (init-interfaces) -- (pull-device-data);
		%
		\draw[arrows.meta] (pull-device-data) -- (eq-mgrs);
		%
		\draw[arrows.meta] (eq-mgrs) -- (subscribe);
		%
		\draw[arrows.meta] (subscribe) -- (pull-gld);
		%
		\draw[arrows.meta] (pull-gld) -- (init-glm-mgr);
		%
		\draw[arrows.meta] (init-glm-mgr) -- (pull-load-data);
		%
		\draw[arrows.meta] (pull-load-data) -- (pull-weather-data);
		%
		\draw[arrows.meta] (pull-weather-data) -- (create-load-models);
		%
		\draw[arrows.meta] (create-load-models) -- (update-glm-mgr);
		%
		\draw[arrows.meta] (update-glm-mgr) -- (init-ga);
		%
		\draw[arrows.meta] (init-ga) -- (run-ga);
		%
		\draw[arrows.meta] (run-ga) -- (send-commands);
		%
		\draw[arrows.meta] (send-commands) -- (check-done);
		%
		\draw[arrows.meta] (check-done) -- (stop);
		%
		\draw[arrows.meta] (check-done.west) -| ++(-4.5,0) |- (update-glm-mgr);
		% Extract/create coordinates for the upcoming optimization phase "fit" box.
		\coordinate (keep-going-west) at ($(check-done.west) + (-4.5,0)$);
		\ExtractCoordinate{$(keep-going-west)$}
		\coordinate (keep-going-mirror) at (\XCoord * -1, \YCoord);
		% Create coordinates for the upcoming initialization phase fit box.
		\ExtractCoordinateTwo{$(init-interfaces)$}
		\coordinate(init-box-west) at (\XCoord, \YCoordTwo);
		\coordinate(init-box-east) at (\XCoord * -1, \YCoordTwo);
	\end{pgfonlayer}

	% Background layer.
	\begin{pgfonlayer}{background}
		% Draw a box around all the initialization data.
		% https://tex.stackexchange.com/a/75498/208656
		\node (init-box) [
			backgroundbox, draw=red, fill=red!20,
			fit= (sim-request) (create-load-models) (init-box-west) (init-box-east)
		] {};
		
		% Draw a box around the optimization routine.
		\node (opt-box) [
			backgroundbox, draw=blue, fill=blue!20,
			fit= (pull-load-data) (send-commands) (check-done) (keep-going-west) (keep-going-mirror)
		] {};
	\end{pgfonlayer}

	% Draw curly brace for initialization.
	\draw [curlybrace]
	(init-box.south east) -- (init-box.north east) node [curlynode] {\Large Initialization Phase};
	
	% Draw curly brace for optimization.
	\draw [curlybrace, decoration={brace, amplitude=40pt, raise=6pt, mirror}]
	(opt-box.south east) -- (opt-box.north east) node [curlynode, xshift=0.65cm] {\Large Optimization Phase};
	
\end{tikzpicture}
\end{document}